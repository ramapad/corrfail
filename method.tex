\section{Detecting correlated dropouts}
\label{sec:method}

When several pinged addresses experience a dropout at the same time,
there may be a common underlying cause. Restated: simultaneous dropouts
could be \emph{dependent} upon each other.

Dependent events are likely to affect addresses that are related to
each other by geography or ISP. We call groups of related addresses, such as all addresses within the
same U.S. state or ISP, \emph{address aggregates}.

\subsection{Finding dependent events in an address aggregate}

When the number of observed addresses ($N$) within an address
aggregate is large, there is some chance that multiple addresses dropout
simultaneously independently (by random chance).

In this section, we describe how we use Binomial hypothesis testing to find events that
are highly unlikely to have occurred independently. Suppose $P_d$ is
the probability that an address from the aggregate drops out in a given window of time. Given
$N$ addresses that can potentially dropout, the Binomial distribution gives the probability that
$K$ of them dropout \emph{independently} as:

\[
{N\choose k} \cdot P_d^k(1-P_d)^{N-k}
\]

Given $N$ and $P_d$, it is possible to calculate $D_{min}$, the minimum number of
dropouts that need to occur, such that the probability of
$D_{min}$ or more dropouts occurring independently is small (less
than 1\%). If we observe $D_{min}$ or more outages, we reason that these dropouts are \emph{not} independent
and that there is likely a common underlying cause behind them.

\subsection{Analyzing $D_{min}$}

Next, we investigate $D_{min}$ values for different combinations of
$N$ and $P_d$ from the Thunderping dataset.

We calculate the probability that an address from an aggregate drops
out in a given window of time, $P_d$, using the Thunderping
dataset as follows:

\begin{equation}
P_d= \frac{total dropouts}{total time bins}
\label{eq:p_dropout}
\end{equation}

We use 11 minute long time bins since Thunderping pings an address
once every 11 minutes from each vantage point. Since dropouts are
detected when every vantage point confirms that its pings have become
unresponsive, a dropout may take up to 11 minutes to get detected. For
each aggregate, the formula calculates the probability that an address
drops out in a 11 minute time bin.

\begin{figure}[t]
\centering
\includegraphics[width=\linewidth]{figs/dropout_rate_660_cdf}
\caption{
\label{fig:p_dropout}
Distribution of $P_d$ for different address aggregates. }
\end{figure}

Figure~\ref{fig:p_dropout} shows how $P_d$ is distributed across
different aggregates. The asn and state-asn curves show large
variation: addresses in some of these aggregates dropout only once every
year whereas in other aggregates, they dropout more often than once per
day.\footnote{Since dropouts are a superset of outages and dynamic
  reassignment, frequent dropouts are not necessarily indicative of
  poor Internet connectivity.} However, when considering entire U.S. states (which contain
several ISPs), $P_d$ exhibits lesser variation. 

$D_{min}$ values depend not only upon $P_d$, but also upon the number
of addresses ($N$) that can potentially fail in an aggregate in a
timebin. 

\begin{figure}[t]
\centering
\includegraphics[width=\linewidth]{figs/upips_per_timebin_660_jan17todec17_cdf}
\caption{
\label{fig:upips_per_timebin}
Distribution of all addresses pinged by Thunderping that can potentially fail in each 11 minute
time bin}
\end{figure}

Figure~\ref{fig:upips_per_timebin} shows the distribution of addresses
pinged by Thunderping in each 11 minute time bin. The median number is
around 50K addresses across all U.S. states and ISPs. Since multiple
weather events tend to be active at a time, the number of addresses
within any aggregate will be strictly smaller. 

For different values of $N$ and $p_d$, we list $D_{min}$ values in
Table~\ref{tbl:binomial_thresh}. 

\begin{table}[th]
%  \scriptsize
  \centering
  \hspace{-0.04in}
  \begin{tabular}{c|c|c|c|c|}
$N$ & \multicolumn{4}{c|}{Probability of dropout} \\
    & \textbf{1/hour} & \textbf{1/day} & \textbf{1/week} &
    \textbf{1/month} \\
    \hline
10 & 6 & 2 & 2 & 1\\
50 & 17 & 3 & 2 & 2\\
100 & 29 & 4 & 2 & 2\\
500 & 113 & 10 & 4 & 2\\
1000 & 213 & 16 & 5 & 3\\
5000 & 982 & 54 & 13 & 5\\
10000 & 1925 & 98 & 20 & 8\\
    \end{tabular}
  \caption{\label{tbl:binomial_thresh} $D_{min}$ values
    for different
    values of number of addresses that can potentially fail ($N$) and
    the probability of dropout ($p_d$). }
\end{table}

These results show that the approach will work for some ASNs (i.e., be
sensitive) but not for others who have such high $p_d$ that we would need to see a very large number of dropouts to suggest that something happened.


% The minimum number of dropouts
%     such that their likelihood of occurrence is small 

% Depending upon 
% Since outages are usually
% rare events, $P_o$ in a small window of time (like minutes) is
% low. Consequently, even a relatively small number of failures


% When that probability is very, very low then we can argue that those X addresses must have failed in a dependent manner, likely due to a common cause.

% Say that there is some probability, p that an outage can occur in any minute. Probability that ‘k’ outages happen independently when we ping ‘n’ addresses is given by the Binomial distribution:
% nCk*(p)^k * (1-p)^n-k

% The main problem is arriving at the probability that an outage occur in any minute. Currently, I divide #outages/total-up-minutes for some aggregate of addresses (say all Comcast addresses in a given US state). Perhaps I can do better. 

% After finding correlated failures, my goal is to make some progress towards the potential cause of each correlated failure. Failures where multiple addresses from multiple ISPs are affected would be consistent with a power outage, for example.

\subsection{Correlated failures span /24s}


\section{Background and related work}
\label{sec:bg}

\subsection{Background: detecting Internet outages using active probes}
% \subsection{Dropouts are supersets of outages and reassignment}

An IP address that is being pinged can be either responsive or
unresponsive. Some addresses never respond to ICMP packets, either
because the address hasn't been assigned to a host or due to ICMP
filtering~\cite{v4-census-imc16} along the path to the host. For such always-unresponsive
addresses, inferring the assigned host's Internet connectivty with ICMP packets is impossible.

For an address that responds to pings, it is possible to use ping
responses to reason about the connectivity of a host to which the address has been
assigned. A ping response from an address indicates that the
host's Internet connection is working. If a previously
responsive address ceases to respond to pings, this could indicate
that the host's Internet connection is no longer working. Recent work
has shown that such an event can also occur as a result of \emph{dynamic
address reassignment}~\cite{addrchange-imc}, when the host's ISP
assigns it a new address and if the old address (the one being pinged)
is no loner assigned to any host, or is assigned to a host that is
configured to not respond to pings.

We define a dropout to be the event where a pinged address \emph{transitions}
from a state where it is responding to probes to a state where it
ceases to respond. Dropouts are therefore supersets of both outages
and dynamic reassignment.

\subsection{Related work}

Large connectivity problems occurring at the Internet's core have
received attention from the community~\cite{censorship-outages, trinocular, hubble, paxson-e2e,
  hubble, netdiagnoser, lifeguard,
  poiroot, phillipa-outages-mailing-list,
  california-fault-lines, delayed-routing-convergence, consensus-routing,
  routing-e2e-path-perf, voip-bgp-convergence}. 

However, at the edge, Internet outages have been studied either by deploying
hardware/software at customer equipment or by conducting active
probing measurement campaigns towards edge hosts from machines under
researcher control. While the former approach allows the detailed study of
a few hosts, the latter enables broader study of more hosts albeit at
lesser detail. 

Disco~\cite{disco} uses Kleinberg's burst detection to detect
events where many RIPE Atlas probes disconnect from
their infrastructure in a correlated manner.

The ``Trinocular'' tool detects outages affecting /24 address
blocks~\cite{trinocular}.  It assumes that addresses in a /24 share fate.
The Zmap tool was used to study outages that occurred during Hurricane Sandy~\cite{durumeric2013zmap}.
Planetseer~\cite{planetseer} uses active probes.
Hubble~\cite{hubble} also uses active probes.

A complementary technique uses large passive data collection to detect
Internet outages at the edge. Dainotti et~al.~\cite{dainotti-imc11} observe Internet Background
Radiation traffic sent to IPv4 darknets to detect outages affecting
entire countries. 

FACT~\cite{fact-flowbased-connectivity} seems related.




\subsubsection{Trinocular detects failures of /24 address blocks}

Trinocular pings addresses in ~4M /24 address blocks and
uses the responses to detect Internet outages affecting entire blocks. Using historical
data from the ISI census~\cite{}, it models the responsiveness of
blocks and finds subsets of addresses within each block that are
likely to respond to pings. The system pings a few of these addresses
from each block at random and probes them in 11-minute
rounds. Trinocular then employs Bayesian inference to reason about
responses from blocks. When a block's responsiveness is lower than
expected, Trinocular probes the block at a faster rate and eventually
detects an outage when the follow-up probes also indicate the block's
lack of Internet connectivity.

The key difference of this work from Trinocular is that we do not assume
that correlated failures span entire /24-address blocks; instead, we
look for correlated failures of addresses related by geography and
ISP. While we share Trinocular's intuiton that dependent events will
affect related addresses, Trinocular's notion of relatedness is solely
that of belonging to the same /24. With the IPv4 address space
breaking up, we hypothesized that addresses in disparate /24s may be
affected by a correlated failure event. Further, a power outage may
affect a few addresses but from several different ISPs.

\subsection{Detecting dependent events using Thunderping}

To evaluate if outages indeed span an entire /24, we use data collected with the
Thunderping~\cite{schulman-imc11} technique, since Thunderping pings
sampled addresses from multiple ISPs in geographic areas in the United
States. Originally designed to evaluate how weather affects Internet
outages, the system uses Planetlab vantage points to ping 100 IPv4
addresses from multiple ISPs in U.S. counties with active
weather alerts. Each address is pinged from multiple Planetlab vantage
points (at least 3) every 11 minutes, and addresses in a county are
pinged six hours before, during, and after a weather alert for that
county. 

We analyze ping responses to detect dropout events for each probed
address. When an address that is responsive stops responding to pings
from all vantage points that are currently probing it, we detect a
dropout event for that address. Since we probe once every 11-minutes
from each vantage point, 
once a vantage point lists an address as unresponsive, we wait for 11 minutes and ensure that
all vantage points that had pinged the address during this period also
confirm its unresponsiveness. Once confirmed, we detect a dropout and
adjust its start-time of the dropout to be the time at which the first
vantage point observed the address to be unresponsive. 

We continue to probe an address after it has become unresponsive,
allowing us to estimate how long the unresponsive period lasted.

% Address why we need Binomial instead of Bayesian. Such a large
% aggregate that Bayesian would only allow entire aggregate
% failures. But Binomial uses correlation in time to detect dropouts
% that happen close together in time. So it's all about the time. And
% even with the coarse granularity of Thunderping, dropouts are rare
% enough that it allows detection of correlated failures often.

% Our position shoud be a motivation for a broader scheme. Need to ping
% more addresses. It's worth it. That should be the goal of this paper.

% When using the Thunderping technique to detect Internet outages affecting
% multiple users, there is another challenge: the Thunderping technique
% will include events where users voluntarily power down their home
% Internet equipment. Since Thunderping probes individual IP addresses, if a user
% powers down their home router, the address assigned to the router will
% cease to respond to pings and this will be detected as a dropout event.

\begin{abstract}

Even in a world that increasingly relies upon the Internet, Internet
failure events, often affecting multiple users, are
known to occur. Characterizing, and quantifying such events is
essential to understanding and improving Internet reliability.

In this paper, we develop and evaluate an approach to detect Internet
failure events that affect multiple users simultaneously using
measurements gathered with the Thunderping technique. Thunderping
pings addresses across U.S. countries and ISPs with severe
weather alerts. It detects a dropout event when an IP address that is responding to pings 
ceases to respond. Our insight is that simultaneous
dropouts of multiple addresses, that are related to each other
by geography or ISP, is indicative of a larger
event. Using a novel approach based upon Binomial hypothesis testing,
we detect instances of events that are unlikely to have
happened by random chance and flag these as candidate Internet failure events.

We characterize Internet failure events and present results
that challenge conventional wisdom on how such outages affect Internet
address blocks. For example, we see that for X\% of such events
affecting at least Y addresses, Z\% had other addresses in an address
block that continued to be responsive to probes. 

We then go
on to segregate failure events into categories that suggest their
cause. Outages could result from a variety of causes, such as power
outages, network
outages due to an ISP's infrastructure failure, cable cuts etc. 

\end{abstract}
